\documentclass[modern]{aastex62}

% TODO
% ----

% Style guidelines
% ----------------
% - Use the Makefile; don't be typing ``pdflatex'' or some bullshit.
% - Wrap text to 80 character lines.
% - Line break between sentences to make the git diffs readable.
% - Use \, as a multiply operator.
% - Reserve () for function arguments; use [] or {} for outer shit.
% - Use \sectionname not Section, \figname not Figure, \documentname not Article

\include{gitstuff}
% Load common packages

\usepackage{amsmath}
\usepackage{amsfonts}
\usepackage{amssymb}
\usepackage{booktabs}

\usepackage{graphicx}
\usepackage{color}

\usepackage{hyperref}
\definecolor{niceblue}{rgb}{0.0, 0.4, 0.65}
\definecolor{linkcolor}{rgb}{0.02,0.35,0.55}
\definecolor{citecolor}{rgb}{0.4,0.4,0.4}
\hypersetup{colorlinks=true,linkcolor=linkcolor,citecolor=citecolor,
            filecolor=linkcolor,urlcolor=linkcolor}
\hypersetup{pageanchor=false}

\newcommand{\documentname}{\textsl{Article}}
\newcommand{\sectionname}{Section}
\newcommand{\figname}{Figure}
\newcommand{\eqname}{Equation}
\newcommand{\tblname}{Table}

% Packages / projects / programming
\newcommand{\package}[1]{\texttt{#1}}
\newcommand{\acronym}[1]{{\small{#1}}}
\newcommand{\github}{\package{GitHub}}
\newcommand{\python}{\package{Python}}

% Missions
\newcommand{\project}[1]{\textsl{#1}}

% For referee
\newcommand{\changes}[1]{{\color{red} #1}}

% Stats / probability
\newcommand{\given}{\,|\,}
\newcommand{\norm}{\mathcal{N}}

% Maths
\newcommand{\dd}{\mathrm{d}}
\newcommand{\transpose}[1]{{#1}^{\!\mathsf{T}}}
\newcommand{\inverse}[1]{{#1}^{-1}}
\newcommand{\argmin}{\operatornamewithlimits{argmin}}
\newcommand{\argmax}{\operatornamewithlimits{argmax}}
\newcommand{\mean}[1]{\left< #1 \right>}
\renewcommand{\vec}[1]{\bs{#1}}
\newcommand{\mat}[1]{\mathbf{#1}}

% Unit shortcuts
\newcommand{\msun}{\ensuremath{\mathrm{M}_\odot}}
\newcommand{\kms}{\ensuremath{\mathrm{km}~\mathrm{s}^{-1}}}
\newcommand{\au}{\ensuremath{\mathrm{au}}}
\newcommand{\pc}{\ensuremath{\mathrm{pc}}}
\newcommand{\kpc}{\ensuremath{\mathrm{kpc}}}
\newcommand{\kmskpc}{\ensuremath{\mathrm{km}~\mathrm{s}^{-1}~\mathrm{kpc}^{-1}}}

% Misc.
\newcommand{\bs}[1]{\boldsymbol{#1}}

% Astronomy
\newcommand{\DM}{{\rm DM}}
\newcommand{\feh}{\ensuremath{{[{\rm Fe}/{\rm H}]}}}
\newcommand{\df}{\acronym{DF}}

% TO DO
\newcommand{\todo}[1]{{\color{red} TODO: #1}}

\graphicspath{{figures/}}

\newcommand{\gaia}{\project{Gaia}}

% Extra packages
\usepackage{microtype}
\usepackage{multirow}
\usepackage{makecell}

% Change the citation color or link color here:
% \definecolor{linkcolor}{rgb}{0.02,0.35,0.55}
% \definecolor{citecolor}{rgb}{0.4,0.4,0.4}

\begin{document}\sloppy\sloppypar\raggedbottom\frenchspacing % trust me

\title{Gaia confirms a disk origin for distant, low-latitude stellar halo structures}

\author[0000-0003-0872-7098]{Adrian~M.~Price-Whelan}
\affiliation{Department of Astrophysical Sciences,
             Princeton University, Princeton, NJ 08544, USA}

% \author[0000-0002-5151-0006]{David~N.~Spergel}
% \affiliation{Flatiron Institute,
%              Simons Foundation,
%              162 Fifth Avenue,
%              New York, NY 10010, USA}
% \affiliation{Department of Astrophysical Sciences,
%              Princeton University, Princeton, NJ 08544, USA}

\correspondingauthor{Adrian M. Price-Whelan}
\email{adrn@astro.princeton.edu}

\begin{abstract}

Several known stellar streams and over-densities in the Milky Way halo --- such
as the so-called Triangulum-Andromeda, Monoceros ring, Galactic Anticenter
Stellar Structure, and A13 --- have relatively low Galactic latitudes,
relatively metal-rich stellar populations, and disk-like chemical abundance
patterns.
These results have therefore been used to suggest that the stars in these
features originated in the Milky Way disk and were ``kicked out'' through
dynamical perturbations from satellite galaxies.
However, their extreme heights above the Galactic mid-plane (up to $\sim
10~\kpc$) have raised questions about the plausibility of this scenario.
Using kinematic measurements from the \gaia\ mission, we show that giant stars
previously identified as probable members of these structures have disk-like
kinematics.
This confirms the origin of these features as having come from the Galactic
disk and suggests...

\end{abstract}

\keywords{
    huh
    ---
    whatever
}

\section{Introduction}
\label{sec:introduction}

The stellar halo around the Milky Way is full of spatial and kinematic
substructures \citep[e.g.,][]{Grillmair:2016}.
The discovery of the vast majority of these structures was enabled by
wide-field, multi-band photometry surveys like the Sloan Digital Sky Survey
\cite[SDSS;][]{York:2000, Gunn:2006, Eisenstein:2011}, Two Micron All-Sky Survey
\citep[2MASS;][]{Skrutskie:2006}, Pan-STARRS \citep[PS1;][]{Chambers:2016}, and
the Dark Energy Survey \citep[DES;][]{Dark-Energy-Survey-Collaboration:2016}.
For example, the long tidal tails of known globular clusters such as the Palomar
5 stream \citep{Odenkirchen:2001}, long, thin structures such as the GD-1 stream
\citep{Grillmair:2006}, and dwarf galaxy streams such as the Orphan stream
\citep{Grillmair:2006a} were all identified using some combination of these
surveys.
Detailed maps of stellar number counts from these surveys have also revealed the
existence of amorphous stellar structures that typically span hundreds of square
degrees on the sky.
Subsequent follow-up of such structures, for example the Triangulum-Andromeda
stellar clouds \citep[TriAnd;][]{Rocha-Pinto:2004, Martin:2007, Sheffield:2014},
the Monoceros ring or Galactic Anticenter Stellar Structure
\citep[Mon/GASS;][]{Newberg:2002, Yanny:2003, Crane:2003, Li:2012,
Morganson:2016}, and A13 \citep{Sharma:2010, Li:2017}, has revealed that stars
in these more diffuse structures tend to have similar metallicities and display
coherent but higher dispersion internal kinematics as inferred from
line-of-sight velocities.
From their observed properties (mainly their spatial distribution on the sky),
Milky Way halo substructures are therefore typically classified into two broad
morphological classes: ``streams'' --- long, relatively thin structures that
exhibit coherent motion --- or ``blobs'' --- physically amorphous, less
kinematically-coherent stellar over-densities.
Importantly, however, recent evidence has shown that these classes do not
uniquely map on to formation scenarios.

Most of the known stellar streams are characterized as having metal poor stellar
populations and are thus likely remnants of fully-disrupted or currently
disrupting dwarf galaxies and globular clusters.
Likewise, some diffuse stellar structures at high Galactic latitudes --- e.g.,
the Hercules-Aquila cloud \citep{Belokurov:2007, Simion:2014} or the Virgo
over-density \citep{Vivas:2001, Carlin:2012} --- are metal poor and may be
``shell''-like features that are seen around external galaxies
\citep[e.g.,][]{Kado-Fong:2018}, though here viewed from the inside.
The joint existence of these stellar substructures and bound stellar satellites
in the Galactic halo therefore appear to confirm the hierarchical nature of
galaxy growth, an important test of cosmological models \citep{Bullock:2005,
Johnston:2008}.

On the other hand, several lines of evidence have recently suggested that a
number of low-latitude stream- and blob-like features may have instead
originated from the Galactic disk rather than from the stellar content of an
accreted system:
\begin{description}
    \item[Spatial distribution]
        Along some sight-lines toward the outer galaxy, Mon/GASS and other
        structures appear to display an oscillatory pattern, over-dense above
        the Galactic plane at some distances and over-dense below the plane at
        others \citep[e.g.,][]{Newberg:2002, Xu:2015}.
        Stream-like features such as the Eastern Banded Structure
        \citep[EBS;][]{Grillmair:2006b} and the Anti-Center Stream
        \citep[ACS;][]{Grillmair:2006b} visually appear to emanate from the
        Galactic disk \citep{Slater:2014, Morganson:2016}.
        However, these structures reach scale heights from $\sim 1~\kpc$
        (Mon/GASS}) to $\sim 10~\kpc$ (TriAnd) above or below the Galactic
        mid-plane.
    \item[Stellar populations]
        TriAnd, Mon/GASS, and A13 all appear to have relatively metal-rich
        stellar populations as compared to known dwarf galaxies, as evidenced by
        their paucity of RR Lyrae-type stars but abundance of more metal-rich
        giant stars \citep{Price-Whelan:2015, Sheffield:2018}.
    \item[Chemical abundances]
        Detailed chemical abundance measurements of giant stars in TriAnd and
        A13 have shown that the abundance patterns of these structures are more
        consistent with an extrapolation of the outer Milky Way disk rather than
        any of the known dwarf galaxies \citep{Bergemann:2018, Hayes:2018}.
    \item[Kinematics]
        TriAnd, Mon/GASS, and A13 seem to form a single kinematic sequence in
        line-of-sight velocity as a function of Galactic longitude
        \citep{Li:2017, Johnston:2017}.
        The 3D kinematics of ACS and Mon/GASS appear to be consistent with the
        disk rotation but move slightly slower than the circular velocity at the
        solar radius and slightly faster than typical thick disk stars.
        Full-space motions of stars in TriAnd and A13 have not yet been studied.
\end{description}

While no coherent picture for the connections or origins of these structures
exists, it is unlikely that they formed from secular evolution of the Galactic
disk alone:
To perturb the outer disk to the extreme scale heights of these observed
structures likely requires external perturbations.
Simulations of gravitational interactions between a thin, self-gravitating disk
and satellite galaxies on orbits similar to Sagittarius and the Magellanic
Clouds have qualitatively demonstrated that diffuse features like Mon/GASS, A13,
and TriAnd and stream-like features such as ACS and EBS can all be produced
through tidal interactions \citep{Purcell:2011, Price-Whelan:2015, Gomez:2016,
Laporte:2018, Laporte:2018a}.
These simulations then predict that all of these structures should have 3D
kinematics and chemical abundances consistent with having been perturbed away
from the Galactic disk.
A holistic model for the formation of large scale height stellar structures such
as ACS, EBS, Mon/GASS, A13, and TriAnd and recently discovered asymmetries in
the inner disk \citep[e.g.,][]{Widrow:2012, Antoja:2018} is therefore of great
interest:
The prospect of using disk stars to study ``Galactoseismology'' combined with
modeling individual outer disk structures will lead to a comprehensive ... of
the interaction history of the Milky Way and a map of dark matter properties
throughout the inner Galactic halo \citep[e.g.,][]{Widrow:2012, Laporte:2018b}.

In this work, we combine astrometry from \gaia\ with previously measured
line-of-sight velocities to study the 3D kinematics of giant stars associated
with the TriAnd and A13 structures...


In \sectionname~\ref{sec:methods} we ...

% \begin{center}
%   \begin{tabular}{ r | c | c | c }
%     \Xhline{2pt}
%     \hspace{0em} & Stellar population & Chemistry & Kinematics\\
%      \Xhline{1pt}
%     EBS      & & & \\\Xhline{0.1pt}
%     ACS      & & & \\\Xhline{0.1pt}
%     Mon/GASS & \checkmark & & \checkmark \\\Xhline{0.1pt}
%     A13      & \checkmark & \checkmark & \\\Xhline{0.1pt}
%     TriAnd   & \checkmark & \checkmark & \\
%     \Xhline{2pt}
%   \end{tabular}
% \end{center}


\section{Methods}
\label{sec:methods}

\section{Data}
\label{sec:data}

\section{Discussion}
\label{sec:discussion}

Guillaume Thomas:
% https://ui.adsabs.harvard.edu/#abs/2019MNRAS.483.3119T/abstract

This thing:
% https://ui.adsabs.harvard.edu/#abs/2018arXiv180707269F/abstract

Xu et al. paper

\acknowledgements

It is a pleasure to thank ...

\software{
The code used in this project is available from \url{GITHUB URL} under the MIT
open-source software license.
This research utilized the following open-source \python\ packages:
    \package{Astropy} \citep{Astropy-Collaboration:2013},
    % \package{emcee} \citep{Foreman-Mackey:2013ascl},
    \package{IPython} \citep{Perez:2007},
    \package{matplotlib} \citep{Hunter:2007},
    \package{numpy} \citep{Van-der-Walt:2011},
    % \package{scipy} (\url{https://www.scipy.org/}),
    % \package{sqlalchemy} (\url{https://www.sqlalchemy.org/}).
% This work additionally used the Gaia science archive
% (\url{https://gea.esac.esa.int/archive/}), and the SIMBAD database
% \citep{Wenger:2000}).
}

% \facility{MDM: Hiltner (Modspec)}

% \appendix

% \section{Some extra stuff} % \label{appdx:}

% \clearpage

\bibliographystyle{aasjournal}
\bibliography{dr2-lls}

\end{document}
